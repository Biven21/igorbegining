Начало КПК с ded Мультик в рамках КПК физтех. \DoxyHorRuler{0}
 Содержание


\begin{DoxyEnumerate}
\item Общее описание сценария.
\item Словарь терминов.
\item Пояснение функций. \DoxyHorRuler{0}

\end{DoxyEnumerate}
\begin{DoxyEnumerate}
\item Общее описания сценария
\end{DoxyEnumerate}

Начала программирования. Начинаем с рисования объектов посредством линий, простых фигур, таких как -\/ окружность, квадрат.

Рисование производим с помощью пакета \char`\"{}Тупой художник\char`\"{} TXLib.

Множество строчек кода, которыми рисуется объект объединяются в функцию, которая в свою очередь имеет параметры, влияющие на координаты характерных точек объекта т.\+е. на размеры объекта и взаимное расположение частей объекта.

С помощью цикла while производим анимацию -\/ движение объектов, а также множественное копирование.

Сюжет фильма навеяло из космоса. 12 апреля 60 лет полета человека в космос. И тяга человечество к колонизации других планет.

Фильм состоит из сцен. Сцена первая\+: \char`\"{}\+Serenity\char`\"{} -\/ безмятежность показывает безымянную планету. все объекты нарисованы с помощью функции POLYGON это позволяет трансформировать объекты и не беспокоиться, что собъются цвета объектов. пример кода -\/ гора центральная\+: \DoxyHorRuler{0}
 tx\+Set\+Color (TX\+\_\+\+LIGHTGRAY); \begin{DoxyVerb}txSetFillColor (RGB (200, 192, 170));

POINT MountainCentral [13] = {{166, 457}, {439, 225}, {480, 263}, {519, 245}, {561, 264}, {615, 249}, {666, 225},

                              {766, 231}, {928, 446}, {840, 432}, {648, 432}, {438, 443}, {278, 443}};

txPolygon (MountainCentral, 13);
\end{DoxyVerb}
 \DoxyHorRuler{0}




Сцена вторая Высадка дисанта (Disembarkation), включающыя в себя также функцию посадки звездолета (Landskape). Прохождение солнц (их кстати два) показывает, что десант высаживался не одни местные сутки.



Сцена третья

Строительство лагеря (Construction\+Camp).

Здесь колонисты выбирают место строительства лагеря и строят дома.

Сцена четвертая\+: выращивание сада Planting\+Garden, сцена состоит из Garden\+Start -\/ посадка сада, Height\+Garden -\/ роста сада.

И конечно начальные (Start\+Titles) и конечные титры (End\+Titles)

 